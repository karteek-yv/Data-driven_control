%\documentclass{beamer}
\RequirePackage[2020-02-02]{latexrelease}
\documentclass[hyperref={pdfpagemode=FullScreen},aspectratio=169]{beamer}

%    1610: 16:10
%    169: 16:9
%    149: 14:9
%    141: 1.41:1
%    54: 5:4
%    43: 4:3 [default]
%    32: 3:2

\setbeamersize{text margin left=1pt,text margin right=1pt}
\mode<presentation> {
%\setbeamertemplate{caption}[numbered]
\setbeamertemplate{footline}[page number]
\setbeamertemplate{bibliography item}{\insertbiblabel}
\usetheme{default}
\usefonttheme{serif}
}
\usepackage{amsfonts,epsfig,subcaption,graphicx,tabularx,amsmath,bm,relsize,steinmetz,booktabs,epstopdf}
\usepackage{rotating}
\usepackage{extarrows,mathtools}
\usepackage{amsrefs}
\usepackage[english]{babel}
\usepackage{epstopdf}
\usepackage{listings}
\usepackage{color}
\usepackage{mathrsfs}
\usepackage{tcolorbox}

\usepackage[nameinlink]{cleveref}
\usepackage{aliascnt}
\crefname{equation}{Eqn.}{Eqns.}
\newaliascnt{ineq}{equation}
\aliascntresetthe{ineq}
\crefname{ineq}{Ineq.}{Ineqs.}
\creflabelformat{ineq}{#2{\upshape(#1)}#3} 
\makeatletter
\def\ineq{$$\refstepcounter{ineq}}
\def\endineq{\eqno \hbox{\@eqnnum}$$\@ignoretrue}
\makeatother
\crefname{figure}{Fig.}{Figs.}
\crefname{table}{Table}{Tables}
%\renewcommand{\figurename}{Fig.} %without babel package
\addto\captionsenglish{\renewcommand{\figurename}{Fig.}} %with babel package
%
%\usepackage{amsthm}
%\newtheorem{definition}{Definition}

\usepackage[printwatermark]{xwatermark}
\usepackage{xcolor}
\usepackage{graphicx}
\usepackage{tikz}
\usetikzlibrary{patterns,decorations.pathmorphing,decorations.markings}
\usetikzlibrary{shapes,arrows,positioning,calc}
\usetikzlibrary{patterns,angles,quotes}
\usepackage{circuitikz}

\newsavebox\mybox
\savebox\mybox{\tikz[color=red,opacity=0.1]\node{\small{Dr. Y. V. Karteek, SCEE, IIT Mandi}};}
\newwatermark*[
  allpages,
  angle=0,
  scale=1,
  xpos=-50,
  ypos=-38
]{\usebox\mybox}

%\title{\Large{UEI303: Techniques on Signals and Systems \\~\\ Random Signals}~\\}
%\title{\Large{UEI407: Signals and Systems\\~\\ Random Signals}~\\}
\title{\Large{EE538: Data-driven Control\\~\\}~\\}
\author{\small Dr. Y. ~V. ~Karteek}
 
\date{} 
%\date{\tiny {\today}}
%----------------------------------------------------
\begin{document}
%----------------------------------------------------
\begin{frame}[plain]
          \begin{center}
                    \Large{}
                    \end{center}
               \titlepage
        \end{frame}
%--------------------------------------------------
\begin{frame}[plain]{Introduction}
\begin{itemize}
\setlength\itemsep{1em}
\item Systems and control theory deals with \textit{controller design problem} for physical systems.
\item Obtaining \textit{mathematical model} of the physical system is the first step.
\item Such a model can be of ODE, PDE, difference equations, transfer functions, transfer matrices etc..
\item The mathematical models are usually based on basic physical laws such as conservation laws, Newton's laws, Kirchhoff's laws, etc..
\item An alternative way to obtain a model is to do experiments on physical system, obtain \textit{data} that can be used to find mathematical descriptions.
\item This method is called \textit{system identification}.
\item The desired behaviour require the mathematical model to have certain properties (qualitative or quantitative), forming a design objective.
\end{itemize}
\end{frame}
%--------------------------------------------------
\begin{frame}[plain]{}
\begin{itemize}
\setlength\itemsep{1em}
\item Final step is to design a mathematical model of the controller.
\item The above approach is called \textit{model-based control}.
\item 
\end{itemize}
\end{frame}
%--------------------------------------------------
\begin{frame}[plain]{}
\begin{itemize}
\setlength\itemsep{1em}
\item 
\end{itemize}
\end{frame}
%--------------------------------------------------

\end{document}
